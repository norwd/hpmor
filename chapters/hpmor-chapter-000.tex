\chapter*{Antaŭparolo}
% el http://www.hpmor.com/chapter/1
% Ĉi tio ne estas strikte fikcio kun nur unu punkto de diverĝeco - ekzistas ne nur primara deirpunkto pasintece sed ankaŭ aliaj ŝanĝoj. La plej bona termino kiu mi aŭdi por ĉi tiu fikcio estas “paralela universo”.

La teksto enhavas multaj indicoj: evidenta indicoj, ne-tiom-evidentaj indicoj, vere obskuraj sugestoj kiu mi ŝokitis ke iu legantoj sukcese malkodis, kaj amasaj evidentecoj elmontrita en plena vido. Tio ĉi estas rakonto racionalista; ĝiaj misteroj solveblas, kaj intencitis esti solvitaj.

La strukturo de la rakonto estas kiel seria fikcio, t.e. televidspektaklo kun antaŭfiksita nombro da sezonoj kaj epizodoj, kaj ankaŭ kun totala rakontarko kondukanta al fina konkludo.

% Ĝis ĉapitro 17 la rakonto estis “ĝustiginta” laŭ brita anglo, kaj plia korektito progresas ĉe la r/HPMOR subreditejo. Tio ne havas signifon por la Esperanta traduko.

Ĉio scienco menciita estas vera. Sed bonvolu memori, ke, preter sciencaj aferoj, la pensoj kaj perspektivoj de la roluloj ne nepre samas al la verkisto (aŭ ja la tradukisto). Ne ĉio, kion faras la protagonisto, estas saĝa, kaj la konsiloj, kiujn donas la pli moralagrizaj karakteroj, povas esti nefidindaj aŭ danĝere dutranĉaj.

\chapter*{Enkonduko de la Verkisto}
% el http://www.hpmor.com/chapter/22

\section*{Io, ie, iam, devis okazi alimaniere…}

\begin{itemize}
\item \textsc{Petunia Evanz} edziniĝis kun Mikael Veres, Profesoro pri Biokemio en Oksfordo.
\item \textsc{Hari Ĝajmz Potr-Evanz-Veres} kreskis en domo plena de libroj ĝis la rando. Li iam mordis matematikan instruiston kiu ne sciis, kio estis logaritmo. Li legis \emph{Gödel, Escher, Bach} kaj \emph{Judgment Under Uncertainty: Heuristics and Biases} kaj volumon unu el \emph{The Feynman Lectures on Physics}. Kaj malgraŭ la ŝajnaj timoj de ĉiu kiu renkontis lin, li ne volas fariĝi la venonta Malhela Sinjoro. Li estis kreskigita pli bone ol tio. Li volas malkovri la leĝojn de magio kaj iĝi dio.
\item \textsc{Hermajni Granger} pli bonas ol li en ĉiu klaso, krom rajdado de balailo.
\item \textsc{Drajko Malfoj} estas ekzakte tiel, kiel vi imagus dekunujara knabo se Darto Vadero estus lia karema patro.
\item \textsc{Profesoro Kŭirel} vivas lian dumvivan sonĝon de instrui la Defendo Kontraŭ la Malhelaj Artoj, aŭ kiel li preferas nomi sian klason, Batala Magio. Liaj studentoj ĉiuj scivolas, kio fuŝiĝos kun la Profesoro pri Defendo ĉi-foje.
\item \textsc{Dumbeldor} estas aŭ freneza, aŭ ludas ian tre pli profundan ludon kiu implikis ekbruligi kokon.
\item \textsc{Minerva Mikgonagol} bezonas foriri ien privatan kaj krii por iom da tempo.
\end{itemize}

% \begin{center}
% Prezentante:
%
% \textsc{Hari Potr kaj la Metodoj de Racionaleco}
%
% Vi ne diven's kien ĉi tiu iras.
% \end{center}

\section*{Iom da notoj}
La opinioj de karakteroj en tiu ĉi rakonto ne nepre estas de la verkisto aŭ la tradukisto. Tio, kion (varma!) Hari opinias, estas ofte intencita kiel bona ŝablono por imiti, precipe se Hari pensas pri kiel li povas citi sciencajn studojn por subteni apartan principon. Sed ne ĉio Hari faras aŭ pensas estas bona ideo. Tiu ne funkcius kiel rakonto. Kaj la malpli varmaj karakteroj povas foje doni valorajn lecionojn, sed tiu lecionoj ankaŭ povas esti danĝere dutranĉaj.

Se vi ne vizitis \url{https://hpmor.com}, ne forgesu fari tion iam; alie vi maltrafos la fanarton, kiel lerni ĉion, kion Hari scias, kaj pli.

Se vi ne nur ĝuis tiun ĉi fikcion, sed lernis ion de ĝin, tial mi petas, konsideru blogi aŭ tviti pri ĝin. Tia verko faras nur tiom da bono kiom estas homoj, kiuj legas ĝin.

%  LocalWords:
