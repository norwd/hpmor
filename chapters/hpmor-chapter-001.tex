\chapter{Tago de Tre Malgranda Probableco}

\begin{chapterOpeningQuote}
\noindent
Sub la lunlumo briletas eta arĝenta fragmento, frakcio de linio…

\vspace*{2ex}
(nigraj roboj, falantaj)

\vspace*{2ex}
…sango verŝiĝas je litroj, kaj iu kriegas vorton.
\end{chapterOpeningQuote}

\lettrine{Ĉ}{iu} colo de muro-spaco estas kovrita per libro-ŝrankaĵo.
Ĉiu libroŝranko havas ses bretojn, ĉiu atinganta preskaŭ ĝis la plafono.
Kelkaj librobretoj estas stakataj ĝis la rando de durkovraj libroj: scienco, matematiko, historio, kaj ĉio alia.
Aliaj bretoj havas du tavolojn de paperkovraj sciencfikciaĵoj, kun la malantaŭa tavolo de libroj apogitaj sur malnovaj naztukskatoloj aŭ lignotabuloj, por vidi la malantaŭan tavolon de libroj super la libroj antaŭaj.
Kaj ankoraŭ ne sufiĉas.
Libroj estas superfluantaj sur la tablojn kaj la sofojn kaj faras amasetojn sub la fenestroj.

Jen la salono de la domo, en kiu loĝas la eminenta Profesoro Mikael Veres\nobreakdash-Evanz, kaj sia edzino, S\nobreakdash-ino.~Petunja Evanz\nobreakdash-Veres, kaj siaj adoptita filo, Hari Ĝajmz Potr\nobreakdash-Evanz\nobreakdash-Veres.

Sur la tablo de la salono estas kuŝanta leteron, apud nestampita koverto el flaveta pergameno, adresita al \emph{S\nobreakdash-ro.~H.~Potr} per smeralda-verda inko.

La Profesoro kaj sia edzino parolas akre unu al alia, sed ili ne laŭtas.
La Profesoro opinias, ke kriado estas necivilizita.

“Vi ŝercas,” Mikael diras al Petunja.
Lia tono indikis ke li tre timis ke ŝi estis serioza.

“Mia filino estis sorĉistino,” Petunja ripetis.
Ŝi aspektis timigita, sed staris firme.
“Ŝia edzo estis sorĉisto.”

“Ĉi tio estas absurda!” Mikael diris akre.
“Ili venis al nia geedziĝo—ili vizitis por Kristnasko—”

“Mi diris al ili, ke vi ne sciu,” flustris Petunja.
“Sed estas la vero. Mi vidis aferojn—”

La profesoro rulis la okulojn.
“Kara, mi komprenas, ke vi ne konas la skeptikan literaturon.
Vi eble ne rimarkas, kiel facile estas por trejnita iluziisto falsi la ŝajne neeblan.
Ĉu vi memoras, kiel mi instruis Hari fleksi kulerojn?
Se ŝajnis, ke ili ĉiam povis diveni, kion vi pensis, tio nomiĝas malvarma legado—”

“Ne estis fleksado de kuleroj—”

“Kio estis do?”

Petunja mordis la lipon.
“Mi simple ne povas diri al vi.
Vi kredos, ke mi estas—”
Ŝi glutis.
“Aŭdu.
Mikael.
Mi ne estis—ĉiam tia—”
Ŝi gestis al si, kvazaŭ por indiki sian sveltan formon.
“Lili faris ĉi tio.
Ĉar mi—ĉar mi \emph{petegis} ŝin.
Dum jaroj, mi petegis ŝin.
Lili \emph{ĉiam} estis pli bela ol mi, kaj mi estis…nebonkora al ŝi, pro tio, kaj tiam ŝi akiris \emph{magion}, ĉu vi povas imagi kiel mi sentis min?
Kaj mi \emph{petegis} ŝin uzi iom da tiu magio sur mi, por ke mi estu bela ankaŭ, eĉ se mi ne povus havi ŝian magion, almenaŭ mi povus esti bela.”

Larmoj kolektiĝis en la okuloj de Petunja.

“Kaj Lili diris al mi ne, kaj elpensis la plej ridindajn ekskuzojn, kiel la mondo finiĝus se ŝi estus afabla al sia fratino, aŭ centaŭro diris al ŝi ne—la plej ridindajn aferojn, kaj mi malamis ŝin pro tio.
Kaj kiam mi ĵus diplomiĝis el universitato, mi geamikis kun knabo, Vernon' Dursli, li estis dika kaj li estis la nura knabo kiu dirus al mi.
Kaj volis infanojn, kaj ke lia unua filo nomiĝus Dadli.
Kaj mi pensis al mi, \emph{kia gepatro nomiĝas sia infano Dadli Dursli?}
Estis kvazaŭ mi vidanta mian tutan estontan vivon streĉantan antaŭ mi, kaj mi ne povis elteni ĝin.
Kaj mi skribis al mia fratino kaj diris al ŝi, ke se ŝi ne helpus min, mi preferus simple—”

Petunja ĉesis.

“Ĉiuokaze,” Petunja diris kun malgranda voĉo, “ŝi cedis.
Ŝi diris al mi ke ĝi estis danĝera, kaj mi diris ke mi ne plu zorgis, kaj mi trinkis pocion kaj mi malsanis dum semajnoj, sed kiam mi resaniĝis mia haŭto klariĝis kaj mi finfine plenigis kaj…mi estis bela, oni estis \emph{agrable} al mi,” ŝia voĉo rompis, “kaj poste mi ne plu povis malami mian filinon, ĉefe kiam mi lernis kion ŝian magion alportis ŝin fine—”

“Karulino,” Mikael diris milde, “vi malsaniĝis, vi akiris iom da pezo dum ripozinta en lito, kaj via haŭto klariĝis memstare.
Aŭ esti malsana igis vin ŝanĝi vian dieton—”

“Ŝi estis sorĉistino,” Petunja ripetis.
“Mi vidis tion.”

“Petunja,” Mikael diris.
La ĝeno estis rampanta en sian voĉon.
“Vi \emph{scias} ke tio ne povas esti vera.
Certe mi ne bezonas klarigi kial?”

Petunja tordis siajn manojn.
Ŝi ŝajnis esti ĉe larmoj.
“Mia amo, mi scias ke mi ne povas venki argumentojn kontraŭ vi, sed bonvolu, vi devas fidi min pri ĉi tio—”

“\emph{Paĉjo! Panjo!}”

Ili du ĉesis kaj rigardis Hari kvazaŭ ili forgesis ke estas tria homo en la ĉambro.

Hari profunde enspiris.
“Panjo, \emph{viaj} gepatroj ne havis magion, ĉu ne?”

“Ne,” Petunja diris, aspektanta perplekse.

“Tial neniu en via familio sciis pri magio kiam Lili ricevis sian leteron.
Kiel \emph{ili} konvinkiĝis?”

“Ha…” Petunja diris.
“Ili ne nur sendis leteron.
Ili sendis profesoron de Hogŭartso.
Li—” la okuloj de Petunja ekflugis al Mikael.
“Li montris al ni iom da magio.”

“Tial vi ne devas batali pri ĉi tio,” Hari diris firme.
Esperante kontraŭ espero ke ĉi-foje, nur ĉi-foje, ili aŭskultus al li.
“Se ĝi estas vera, ni povas simple alporti Hogŭartso profesoro ĉi tien kaj vidi la magion por ni mem, kaj Paĉjo akceptos ke ĝi estas vera.
Kaj se ne, tiam Panjo akceptos ke ĝi estas malvera.
Tio estas kio la eksperimenta metodo estas por, do ke ni ne devu solvi aferojn nur per argumentado.”

La profesoro turnis sin kaj rigardis malsupren al li, arogante kiel kutima.
“Ho, estu racia, Hari.
Ĉu vere, \emph{magio}?
Mi pensis ke \emph{vi} scius pli bone ol preni ĉi tion serioze, filo, eĉ se vi estas nur dekjara.
Magio estas proksimume la plej nescienca afero kio estas!”

La buŝo de Hari maldolĉe tordis.
Li estis traktita bone, eble pli bone ol plejparto de genetikaj patroj traktis siajn proprajn infanojn.
Hari estis sendita al la plej bonajn bazlernejojn—kaj kiam tio ne funkciis, al li estis provizitaj gvidinstruistoj de la senfina provizo de malsataj studentoj.
Ĉiam Hari estis kuraĝigita studi kion ajn kaptis lian atenton, por li estis aĉetitaj ĉiuj libroj kiuj kaptis lian fantazion, li estis sponsorinta en kiaj ajn matematikaj aŭ sciencaj konkursoj li eniris.
Li ricevis ion ajn akcepteblan, kion li volis, krom, eble, la plej malgrandan peceton de respekto.
Doktoro instruante biokemion ĉe Oksfordo apenaŭ povus esti atendita por aŭskulti la konsilon de eta knabo.
Vi aŭskultus por ke Montru Intereson, certe; tio estas kio Bona Gepatro farus, kaj do, se vi konceptis vin mem tiel Bona Gepatro, vi farus tiel.
Sed ĉu vi prenas dek-jarulon \emph{serioze}?
Apenaŭ.

Kelkfoje Hari volis krii al sia patro.

“Panjo,” Hari diris.
“Se vi volas venki ĉi tiun argumenton kontraŭ Paĉjo, rigardu la duan ĉapitron de la unua libro de \emph{La Prelegoj de Fajnman pri Fiziko}.
Estas citaĵo tie pri kiel filozofoj diras multon pri tio, kion scienco absolute bezonas, kaj tio estas tute erara, ĉar la sola regulo en scienco estas ke la fina arbitro estas observado—ke vi simple devu rigardi la mondon kaj raportu tion, kion vi vidas.
Nu…de la supro de mia kapo, mi ne povas pensi kie onu povas trovi ion pri kiel estas idealo de scienco solvi aferojn per eksperimentoj anstataŭ per argumentoj—”

% His patrino looked down at him kaj smiled.
% “Thank you, Hari.
% But—” her head rose back up to stare at her husband.
% “I don’t want to win an argument with your patro.
% I want my husband to, to listen to his wife who loves him, kaj trust her just this once—”

% Hari closed his eyes briefly.
% \emph{Hopeless.}
% Both of his parents were just hopeless.

% Now his parents were getting into one of \emph{those} arguments again, one where his patrino tried to make his patro feel guilty, kaj his patro tried to make his patrino feel stupid.

% “I’m going to go to my room,” Hari announced.
% His voice trembled a little.
% “Please try not to fight too much about this, Panjo, Paĉjo, we’ll know soon enough how it comes out, right?”

% “Of course, Hari,” said his patro, kaj his patrino gave him a reassuring kiss, kaj then they went on fighting while Hari climbed la stairs to his bedroom.

% He shut la door behind him kaj tried to think.

% La funny thing was, he \emph{should} have agreed with Paĉjo.
% No-one had ever seen any evidence of magic, kaj according to Panjo, there was a whole magical world out there.
% How could anyone keep something like that a secret?
% More magic?
% That seemed like a rather suspicious sort of excuse.

% It should have been a clean case for Panjo joking, lying or being insane, in ascending order of awfulness.
% If Panjo had sent la letter herself, that would explain how it arrived at la letterbox without a stamp.
% A little insanity was far, far less improbable than la universe really working like that.

% Except that some part of Hari was utterly convinced that magic was real, kaj had been since la instant he saw la putative letter from la Hogŭartso School of Witchcraft kaj Wizardry.

% Hari rubbed his forehead, grimacing.
% \emph{Don’t believe everything you think,} one of his books had said.

% But this bizarre certainty…Hari was finding himself just \emph{expecting} that, yes, a Hogŭartso profesoro would show up kaj wave a wand kaj magic would come out.
% La strange certainty was making no effort to guard itself against falsification—wasn’t making excuses in advance for why there wouldn’t be a profesoro, or la profesoro would only be able to bend spoons.

% \emph{Where do you come from, strange little prediction?}
% Hari directed la thought at his brain.
% \emph{Kial mi kredas tion, kion mi kredas?}

% Usually Hari was pretty good at answering that question, but in this particular case, he had no \emph{clue} what his brain was thinking.

% Hari mentally shrugged.
% A flat metal plate on a door affords pushing, kaj a handle on a door affords pulling, kaj la thing to do with a testable hypothesis is to go kaj test it.

% He took a piece of lined paper from his desk, kaj started writing.

\begin{writtenNote}
\letterAddress{Estimata Deputita Lernejestrino}
\end{writtenNote}

Hari paŭzis, reflektante; tiam forĵetis la paperon por alia, frapante alian milimetron da grafito el sia mekanika krajono.
Ĉi tio postulis zorgan kaligrafion.

\begin{writtenNote}
\letterAddress{Estimata Deputita Lernejestrino Minerva Mikgonagol,}

\letterAddress{Aŭ Kiu Ajn Povas Esti Koncerninta:}

Mi ĵus ricevis vian akceptan leteron de Hogŭartso, adresita al S\nobreakdash-ro.~H.~Potr.
Vi eble ne scias ke miaj genetikaj gepatroj, Ĝajmz Potr kaj Lili Potr (antaŭe Lili Evanz) estas mortaj.
Mi estis adoptita per la filino de Lili, Petunja Evanz-Veres, kaj ŝia edzo, Mikael Veres-Evanz.

Mi estas ege interesata por ĉeesti Hogŭartson, kondiĉe de tia loko efektive ekzistanta.
Nur mia patrino Petunja diras ke ŝi scias pri magio, kaj ŝi mem ne povas uzi ĝin.
Mia patro estas tre skeptika.
Mi mem estas malcerta.
Mi ankaŭ ne scias kie akiri iun ajn de la libroj aŭ ekipaĵoj listigitaj en via akcepta letero.

Patrino menciis ke vi sendis Hogŭartsan reprezentanton al Lili Potr (tiam Lili Evanz) por demonstri al ŝia familio ke magio estas reala, kaj, mi supozas, por helpi Lili akiri siajn lernejajn materialojn.
Se vi povus fari tion por mia familio, estus tre helpema.

\letterClosing[Sincere,]{Hari Ĝajmz Potr-Evanz-Veres.}
\end{writtenNote}

% Hari added their current address, then folded up la letter kaj put it in an envelope, which he addressed to Hogŭartso.
% Further consideration led him to obtain a candle kaj drip wax onto la flap of la envelope, into which, using a penknife’s tip, he impressed la initials H.Ĝ.P.E.V\@.
% If he was going to descend into this madness, he was going to do it with style.

% Then he opened his door kaj went back downstairs.
% His patro was sitting in la living-room kaj reading a book of higher maths to show how smart he was; kaj his patrino was in la kitchen preparing one of his patro’s favourite meals to show how loving she was.
% It didn’t look like they were talking to one another at all.
% As scary as arguments could be, \emph{not arguing} was somehow much worse.

% “Panjo,” Hari said into la unnerving silence, “I’m going to test la hypothesis.
% According to your theory, how do I send an owl to Hogŭartso?”

% His patrino turned from la kitchen sink to stare at him, looking shocked.
% “I—I don’t know, I think you just have to own a magic owl.”

% That should’ve sounded highly suspicious, \emph{oh, so there’s no way to test your theory then}, but la peculiar certainty in Hari seemed willing to stick its neck out even further.

% “Well, la letter got here somehow,” Hari said, “so I’ll just wave it around outside kaj call ‘letter for Hogŭartso!’ kaj see if an owl picks it up.
% Paĉjo, do you want to come kaj watch?”

% His patro shook his head minutely kaj kept on reading.
% \emph{Of course,} Hari thought to himself.
% Magic was a disgraceful thing that only stupid people believed in; if his patro went so far as to \emph{test} la hypothesis, or even \emph{watch} it being tested, that would feel like \emph{associating} himself with that…

% Only as Hari stumped out la back door, into la back garden, did it occur to him that if an owl \emph{did} come down kaj snatch la letter, he was going to have some trouble telling Paĉjo about it.

% \emph{But—well—that can’t \emph{really} happen, can it?
% No matter what my brain seems to believe.
% If an owl really comes down kaj grabs this envelope, I’m going to have worries a lot more important than what Paĉjo thinks.}

% Hari took a deep breath, kaj raised la envelope into la air.

% He swallowed.

% Calling out \emph{Letter for Hogŭartso!} while holding an envelope high in la air in la middle of your own back garden was…actually pretty embarrassing, now that he thought about it.

% \emph{No.
% I’m better than Paĉjo.
% I will use la scientific method even if it makes me feel stupid.}

% “Letter—” Hari said, but it actually came out as more of a whispered croak.

% Hari steeled his will, kaj shouted into la empty sky, “\emph{Letter for Hogŭartso!
% Can I get an owl?}”

% “Hari?” asked a bemused woman’s voice, one of la neighbours.

% Hari pulled down his hand like it was on fire kaj hid la envelope behind his back like it was drug money.
% His whole face was hot with shame.

% An old woman’s face peered out from above la neighbouring fence, grizzled grey hair escaping from her hairnet.
% S\nobreakdash-ino.~Figg, la occasional babysitter.
% “What are you doing, Hari?”

% “Nothing,” Hari said in a strangled voice.
% “Just—testing a really silly theory—”

% “Did you get your acceptance letter from Hogŭartso?”

% Hari froze in place.

% “Yes,” Hari’s lips said a little while later.
% “I got a letter from Hogŭartso.
% They say they want my owl by la 31st of July, but—”

% “But you don’t \emph{have} an owl.
% Poor dear!
% I can’t imagine \emph{what} someone must have been thinking, sending you just la standard letter.”

% A wrinkled arm stretched out over la fence, kaj opened an expectant hand.
% Hardly even thinking at this point, Hari gave over his envelope.

% “Just leave it to me, dear,” said S\nobreakdash-ino.~Figg, “and in a jiffy or two I’ll have someone over.”

% Kaj her face disappeared from over la fence.

% There was a long silence in la garden.

Tiam knaba voĉo diris, trankvile kaj kviete, “Kio.”
