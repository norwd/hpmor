\chapter{Tago de Tre Malgranda Probableco}

\begin{chapterOpeningQuote}
\noindent
Sub la lunlumo briletas etan arĝentan fragmenton, frakcion linion…

\vspace*{2ex}
(nigraj roboj, falantaj)

\vspace*{2ex}
…sango verŝiĝas je litroj, kaj iu kriegas vorton.
\end{chapterOpeningQuote}

\lettrine{Ĉ}{iu} colo da murspaco esta kovrita per libroŝrankaĵo. Ĉiu libroŝranko havas ses bretoj, atingantaj preskaŭ ĝis la plafono. Kelkaj librobretoj estas stakitaj ĝis la rando je durkovraj libroj: scienco, matematiko, historio, kaj ĉio alia. Aliaj bretoj havas du tavolojn de paperkovraj sciencfikciaĵoj, kun la malantaŭa tavolo de libroj apogita sur malnovaj naztukskatoloj aŭ lignotabuloj, por ke oni povas vidi la malantaŭan tavolon de libroj super la libroj antaŭaj. Kaj ankoraŭ ne sufiĉas. Libroj superfluantas sur la tablojn kaj la sofojn kaj faras amasetojn sub la fenestroj.

Jen la salono de la domo, en kiu loĝas la eminenta Profesoro Mikael Veres\nobreakdash-Evanz, kaj sia edzino, S\nobreakdash-ino.~Petunja Evanz\nobreakdash-Veres, kaj siaj adoptita filo, Hari Ĝajmz Potr-Evanz-Veres.

Sur la tablo de la salono kuŝantas leteron, apud nestampita koverto el flaveta pergameno, adresita al \emph{S\nobreakdash-ro.~H.~Potr} per smeraldverda inko.

La Profesoro kaj sia edzino parolas akre unu al alia, sed ili ne laŭtas. La Profesoro opinias, ke kriado estas necivilizita.

“Vi ŝercas,” Mikael drias al Petunja. Lia tono indikis ke li tre timis ke ŝi estis serioza.

“Mia filino estis sorĉistino,” Petunja ripetis. Ŝi aspektis timigita, sed staris firme. “Ŝia edzo estis sorĉisto.”

“Ĉi tio estas absurda!” Mikael diris akre. “Ili venis al nia geedziĝo—ili vizitis por Kristnasko—”

“Mi diris al ili, ke vi ne sciu,” flustris Petunja. “Sed ĝi estas vera. Mi vidis aferojn—”

The Professor rolled his eyes. “Dear, I understand that you’re not familiar with the sceptical literature. You may not realise how easy it is for a trained magician to fake the seemingly impossible. Remember how I taught Hari to bend spoons? If it seemed like they could always guess what you were thinking, that’s called cold reading—”

“It wasn’t bending spoons—”

“What was it, then?”

Petunja bit her lip. “I can’t just tell you. You’ll think I’m—” She swallowed. “Listen. Mikael. I wasn’t—always like this—” She gestured at herself, as though to indicate her lithe form. “Lili did this. Because I—because I \emph{begged} her. For years, I begged her. Lili had \emph{always} been prettier than me, and I’d…been mean to her, because of that, and then she got \emph{magic}, can you imagine how I felt? And I \emph{begged} her to use some of that magic on me so that I could be pretty too, even if I couldn’t have her magic, at least I could be pretty.”

Tears were gathering in Petunja’s eyes.

“And Lili would tell me no, and make up the most ridiculous excuses, like the world would end if she were nice to her sister, or a centaur told her not to—the most ridiculous things, and I hated her for it. And when I had just graduated from university, I was going out with this boy, Vernon Dursley, he was fat and he was the only boy who would talk to me. And he said he wanted children, and that his first son would be named Dadli. And I thought to myself, \emph{what kind of parent names their child Dadli Dursli?} It was like I saw my whole future life stretching out in front of me, and I couldn’t stand it. And I wrote to my sister and told her that if she didn’t help me I’d rather just—”

Petunja stopped.

“Anyway,” Petunja said, her voice small, “she gave in. She told me it was dangerous, and I said I didn’t care any more, and I drank this potion and I was sick for weeks, but when I got better my skin cleared up and I finally filled out and…I was beautiful, people were \emph{nice} to me,” her voice broke, “and after that I couldn’t hate my sister any more, especially when I learned what her magic brought her in the end—”

“Darling,” Mikael said gently, “you got sick, you gained some weight while resting in bed, and your skin cleared up on its own. Or being sick made you change your diet—”

“She was a witch,” Petunja repeated. “I saw it.”

“Petunja,” Mikael said. The annoyance was creeping into his voice. “You \emph{know} that can’t be true. Do I really have to explain why?”

Petunja wrung her hands. She seemed to be on the verge of tears. “My love, I know I can’t win arguments with you, but please, you have to trust me on this—”

“\emph{Paĉjo! Panjo!}”

The two of them stopped and looked at Hari as though they’d forgotten there was a third person in the room.

Hari took a deep breath. “Mum, \emph{your} parents didn’t have magic, did they?”

“No,” Petunja said, looking puzzled.

“Then no-one in your family knew about magic when Lili got her letter. How did \emph{they} get convinced?”

“Ah…” Petunja said. “They didn’t just send a letter. They sent a professor from Hogŭartso. He—” Petunja’s eyes flicked to Mikael. “He showed us some magic.”

“Then you don’t have to fight over this,” Hari said firmly. Hoping against hope that this time, just this once, they would listen to him. “If it’s true, we can just get a Hogŭartso professor here and see the magic for ourselves, and Paĉjo will admit that it’s true. And if not, then Panjo will admit that it’s false. That’s what the experimental method is for, so that we don’t have to resolve things just by arguing.”

The Professor turned and looked down at him, dismissive as usual. “Oh, come now, Hari. Really, \emph{magic}? I thought \emph{you’d} know better than to take this seriously, son, even if you’re only ten. Magic is just about the most unscientific thing there is!”

Hari’s mouth twisted bitterly. He was treated well, probably better than most genetic fathers treated their own children. Hari had been sent to the best primary schools—and when that didn’t work out, he was provided with tutors from the endless pool of starving students. Always Hari had been encouraged to study whatever caught his attention, bought all the books that caught his fancy, sponsored in whatever maths or science competitions he entered. He was given anything reasonable that he wanted, except, maybe, the slightest shred of respect. A Doctor teaching biochemistry at Oxford could hardly be expected to listen to the advice of a little boy. You would listen to Show Interest, of course; that’s what a Good Parent would do, and so, if you conceived of yourself as a Good Parent, you would do it. But take a ten-year-old \emph{seriously}? Hardly.

Kelkfoje Hari volis krii al sia patro.

“Panjo,” Hari said. “If you want to win this argument with Paĉjo, look in chapter two of the first book of the Feynman Lectures on Physics. There’s a quote there about how philosophers say a great deal about what science absolutely requires, and it is all wrong, because the only rule in science is that the final arbiter is observation—that you just have to look at the world and report what you see. Um…off the top of my head I can’t think of where to find something about how it’s an ideal of science to settle things by experiment instead of arguments—”

His mother looked down at him and smiled. “Thank you, Hari. But—” her head rose back up to stare at her husband. “I don’t want to win an argument with your father. I want my husband to, to listen to his wife who loves him, and trust her just this once—”

Hari closed his eyes briefly. \emph{Hopeless.} Both of his parents were just hopeless.

Now his parents were getting into one of \emph{those} arguments again, one where his mother tried to make his father feel guilty, and his father tried to make his mother feel stupid.

“I’m going to go to my room,” Hari announced. His voice trembled a little. “Please try not to fight too much about this, Panjo, Paĉjo, we’ll know soon enough how it comes out, right?”

“Of course, Hari,” said his father, and his mother gave him a reassuring kiss, and then they went on fighting while Hari climbed the stairs to his bedroom.

He shut the door behind him and tried to think.

The funny thing was, he \emph{should} have agreed with Paĉjo. No-one had ever seen any evidence of magic, and according to Panjo, there was a whole magical world out there. How could anyone keep something like that a secret? More magic? That seemed like a rather suspicious sort of excuse.

It should have been a clean case for Panjo joking, lying or being insane, in ascending order of awfulness. If Panjo had sent the letter herself, that would explain how it arrived at the letterbox without a stamp. A little insanity was far, far less improbable than the universe really working like that.

Except that some part of Hari was utterly convinced that magic was real, and had been since the instant he saw the putative letter from the Hogŭartso School of Witchcraft and Wizardry.

Hari rubbed his forehead, grimacing. \emph{Don’t believe everything you think,} one of his books had said.

But this bizarre certainty…Hari was finding himself just \emph{expecting} that, yes, a Hogŭartso professor would show up and wave a wand and magic would come out. The strange certainty was making no effort to guard itself against falsification—wasn’t making excuses in advance for why there wouldn’t be a professor, or the professor would only be able to bend spoons.

\emph{Where do you come from, strange little prediction?} Hari directed the thought at his brain. \emph{Why do I believe what I believe?}

Usually Hari was pretty good at answering that question, but in this particular case, he had no \emph{clue} what his brain was thinking.

Hari mentally shrugged. A flat metal plate on a door affords pushing, and a handle on a door affords pulling, and the thing to do with a testable hypothesis is to go and test it.

He took a piece of lined paper from his desk, and started writing.

\begin{writtenNote}
\letterAddress{Estimata Deputita Lernejestrino}
\end{writtenNote}

Hari paused, reflecting; then discarded the paper for another, tapping another millimetre of graphite from his mechanical pencil. This called for careful calligraphy.

\begin{writtenNote}
\letterAddress{Estimata Deputita Lernejestrino Minerva Mikgonagol,}

\letterAddress{Aŭ Kiu Ajn Povas Esti Koncernita:}

I recently received your letter of acceptance to Hogŭartso, addressed to S\nobreakdash-ro.~H.~Potr. You may not be aware that my genetic parents, Ĝajmz Potr and Lili Potr (formerly Lili Evanz) are dead. I was adopted by Lili’s sister, Petunja Evanz-Veres, and her husband, Mikael Veres-Evanz.

I am extremely interested in attending Hogŭartso, conditional on such a place actually existing. Only my mother Petunja says she knows about magic, and she can’t use it herself. My father is highly sceptical. I myself am uncertain. I also don’t know where to obtain any of the books or equipment listed in your acceptance letter.

Mother mentioned that you sent a Hogŭartso representative to Lili Potr (then Lili Evanz) in order to demonstrate to her family that magic was real, and, I presume, help Lili obtain her school materials. If you could do this for my own family it would be extremely helpful.

\letterClosing[Sincerely,]{Hari Ĝajmz Potr-Evanz-Veres.}
\end{writtenNote}

Hari added their current address, then folded up the letter and put it in an envelope, which he addressed to Hogŭartso. Further consideration led him to obtain a candle and drip wax onto the flap of the envelope, into which, using a penknife’s tip, he impressed the initials H.Ĝ.P.E.V\@. If he was going to descend into this madness, he was going to do it with style.

Then he opened his door and went back downstairs. His father was sitting in the living-room and reading a book of higher maths to show how smart he was; and his mother was in the kitchen preparing one of his father’s favourite meals to show how loving she was. It didn’t look like they were talking to one another at all. As scary as arguments could be, \emph{not arguing} was somehow much worse.

“Panjo,” Hari said into the unnerving silence, “I’m going to test the hypothesis. According to your theory, how do I send an owl to Hogŭartso?”

His mother turned from the kitchen sink to stare at him, looking shocked. “I—I don’t know, I think you just have to own a magic owl.”

That should’ve sounded highly suspicious, \emph{oh, so there’s no way to test your theory then}, but the peculiar certainty in Hari seemed willing to stick its neck out even further.

“Well, the letter got here somehow,” Hari said, “so I’ll just wave it around outside and call ‘letter for Hogŭartso!’ and see if an owl picks it up. Paĉjo, do you want to come and watch?”

His father shook his head minutely and kept on reading. \emph{Of course,} Hari thought to himself. Magic was a disgraceful thing that only stupid people believed in; if his father went so far as to \emph{test} the hypothesis, or even \emph{watch} it being tested, that would feel like \emph{associating} himself with that…

Only as Hari stumped out the back door, into the back garden, did it occur to him that if an owl \emph{did} come down and snatch the letter, he was going to have some trouble telling Paĉjo about it.

\emph{But—well—that can’t \emph{really} happen, can it? No matter what my brain seems to believe. If an owl really comes down and grabs this envelope, I’m going to have worries a lot more important than what Paĉjo thinks.}

Hari took a deep breath, and raised the envelope into the air.

He swallowed.

Calling out \emph{Letter for Hogŭartso!} while holding an envelope high in the air in the middle of your own back garden was…actually pretty embarrassing, now that he thought about it.

\emph{No. I’m better than Paĉjo. I will use the scientific method even if it makes me feel stupid.}

“Letter—” Hari said, but it actually came out as more of a whispered croak.

Hari steeled his will, and shouted into the empty sky, “\emph{Letter for Hogŭartso! Can I get an owl?}”

“Hari?” asked a bemused woman’s voice, one of the neighbours.

Hari pulled down his hand like it was on fire and hid the envelope behind his back like it was drug money. His whole face was hot with shame.

An old woman’s face peered out from above the neighbouring fence, grizzled grey hair escaping from her hairnet. S\nobreakdash-ino.~Figg, the occasional babysitter. “What are you doing, Hari?”

“Nothing,” Hari said in a strangled voice. “Just—testing a really silly theory—”

“Did you get your acceptance letter from Hogŭartso?”

Hari froze in place.

“Yes,” Hari’s lips said a little while later. “I got a letter from Hogŭartso. They say they want my owl by the 31st of July, but—”

“But you don’t \emph{have} an owl. Poor dear! I can’t imagine \emph{what} someone must have been thinking, sending you just the standard letter.”

A wrinkled arm stretched out over the fence, and opened an expectant hand. Hardly even thinking at this point, Hari gave over his envelope.

“Just leave it to me, dear,” said S\nobreakdash-ino.~Figg, “and in a jiffy or two I’ll have someone over.”

And her face disappeared from over the fence.

There was a long silence in the garden.

Tiam knaba voĉo diris, trankvile kaj kviete, “Kio.”
