\definecolor{gold}{rgb}{0.77,0.69,0.37}
\newlength{\hptitlewidth}
\newlength{\rationalh}
\newcommand{\hptitle}[2][\stockwidth]{%
  \setlength{\hptitlewidth}{#1}%
  \centering\color{white}%
  \vskip 3cm\resizebox{.95\hptitlewidth}{!}{\textls[100]{HARI POTR KAJ LA}}%
  \vskip 2mm%
  \color{gold}%
  \settoheight{\rationalh}{\resizebox{.95\hptitlewidth}{!}{\textls[20]{RACIONALECO}}}
  \resizebox{!}{\rationalh}{\textls[50]{METODOJ}}%
  \hfil\resizebox{!}{\rationalh}{\textls[50]{DE}}%
  \vskip 2mm%
  \resizebox{.95\hptitlewidth}{!}{\textls[20]{RACIONALECO}}%
  \vskip 8mm%
  \color{white}%
  \resizebox{.5\hptitlewidth}{!}{\textls[50]{\scshape{}verkita de Eliezer Judkaŭski}}%
  \vfill%
  \textls[50]{\scshape #2}%
  \color{black}%
  \vskip 1cm\ %
}
\providecommand{\fullvolumetitle}[1]{Libro #1: \volumetitle}

\ifcover%
\newpagecolor{black}\afterpage{\restorepagecolor}
\newcommand\BackgroundPic{
\put(0,0){%
\parbox[b][\paperheight]{\paperwidth}{%
\vfill%
\centering%
\includegraphics[width=\paperwidth,height=\paperheight,keepaspectratio]{images/cover0.jpg}%
\vfill%
}}}\AddToShipoutPicture*{\BackgroundPic}%
\AddToShipoutPicture*{\put(0,0){%
\parbox[b][\paperheight]{\paperwidth}{%
\hptitle{\fullvolumetitle{\volumenumber}}%
}}}%
\ %
\cleartorecto
\fi
\begin{center}
\thispagestyle{empty}
{\hpfont
\Huge\MakeUppercase{Hari Potr}\vspace*{0.5cm}

\Large\MakeUppercase{kaj la Metodoj de Racionaleco} %

\includegraphics[scale=0.5]{images/bubble0.jpg}

\Large VERKITA DE \vspace*{.25cm}

\huge \MakeUppercase{Eliezer Judkaŭski}%

\normalsize

\vspace*{1\baselineskip}
\fullvolumetitle{\volumenumber}
}

\vfill
Trovu la originalan tekston, kun la notoj de la verkisto, fanartoj, kaj aliaj informoj ĉe: \url{http://hpmor.com}

La hejmo de ĉi tiu tradukita bitlibro versio estas: \url{https://github.com/norwd/hpmor/}
\end{center}

\cleartoverso

\begin{center}
\vspace*{2cm}

\thispagestyle{empty}
Fanfikcio bazita sur la karakteroj de

\vspace*{.5cm}

\Large \MakeUppercase{J.~K.~Roŭling} \normalsize

\vspace*{.5cm}

kaj siaj libroj:

\vspace*{.5cm}

{
        \newcounter{books_list_counter}
        \def \hpBook #1{
                \addtocounter{books_list_counter}{1}
                \textit{Hari Potr kaj la #1} \par
                \the\value{books_list_counter}a jaro ĉe Hogŭartso
                \smallskip\par
        }
        \hpBook{Ŝtono de la Filozofo}
        \hpBook{Ĉambro de Sekretoj}
        \hpBook{Malliberulo de Azkabano}
        \hpBook{Kaliko de Fajro}
        \hpBook{Ordeno de la Fenikso}
        \hpBook{Duonsanga Princo}
        \hpBook{Sanktaĵoj de Morto}
}
\end{center}
\cleartorecto% FIXME: For some reason, without this the contents ends up on a verso page (an extra blank page is added)

% \chapter{Senrespondecigo}
\newpage
\vspace*{4cm}
\begin{center}
Senrespondecigo:\\J.~K.~Roŭling posedas Hari Potr,\\kaj neniu posedas la metodojn de racionaleco.
\end{center}

\vfill
Ĉi tiu bitlibro estis kreita \today{}.
