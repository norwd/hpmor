% This file includes all the generic formatting for HPatMoR. This mostly entails configuring
% the memoir package, though “configuring” on occasion means “completely messing it up”.

\RequirePackage{fmtcount}
\RequirePackage{calc}

% Fonts used generally (specific fonts used only once or twice are not here).
\usepackage{xltxtra}
\defaultfontfeatures{Ligatures={TeX}}
\setmainfont[
  Extension=.otf
, UprightFont=*-Regular
, ItalicFont=*-Italic
, BoldFont=*-Bold
, BoldItalicFont=*-BoldItalic
, SmallCapsFont=AlegreyaSC-Regular
]{Alegreya}

\newfontface\hpfont[ExternalLocation]{Lumos}
\newfontface\ptsansi[ExternalLocation]{AlegreyaSans-Italic}

% Drop-caps at start of chapters
\renewcommand{\LettrineFontHook}{\hpfont}
\renewcommand{\LettrineTextFont}{}
\renewcommand{\DefaultLoversize}{.2}
\renewcommand{\DefaultLraise}{0.1}

% Extend lettrine cutout when more than one paragraph goes alongside the drop-cap
% Copied, with different macro names, from
% https://tex.stackexchange.com/questions/369868/using-lettrine-with-short-paragraphs
\newcount\hplettrinecount
\makeatletter
\def\hplettrineextrapara{%
\ifnum\prevgraf<\c@L@lines
\hplettrinecount\z@
\loop
\ifnum\hplettrinecount<\prevgraf
\advance\hplettrinecount\@ne
\afterassignment\hplettrinedima\count@\L@parshape\relax
\repeat
\parshape\L@parshape
\fi}
\def\hplettrinedima{\afterassignment\hplettrinedimb\dimen@}
\def\hplettrinedimb{\afterassignment\hplettrinedef\dimen@}
\def\hplettrinedef#1\relax{\edef\L@parshape{\the\numexpr\count@-1\relax\space #1}}
\makeatother
\newcommand{\lettrinepara}[3][]{\lettrine[nindent=0pt,#1]{#2}{#3}}

% Allow linebreaks after em-dash and hyphens, except when they’re followed by punctuation
\newXeTeXintercharclass \punctuationClass

\XeTeXcharclass `\’ \punctuationClass
\XeTeXcharclass `\‘ \punctuationClass
\XeTeXcharclass `\“ \punctuationClass
\XeTeXcharclass `\” \punctuationClass
\XeTeXcharclass `\. \punctuationClass
\XeTeXcharclass `\, \punctuationClass
\XeTeXcharclass `\: \punctuationClass
\XeTeXcharclass `\? \punctuationClass
\XeTeXcharclass `\! \punctuationClass
\XeTeXcharclass `\: \punctuationClass

\newXeTeXintercharclass \digitClass
\XeTeXcharclass `\0 \digitClass
\XeTeXcharclass `\1 \digitClass
\XeTeXcharclass `\2 \digitClass
\XeTeXcharclass `\3 \digitClass
\XeTeXcharclass `\4 \digitClass
\XeTeXcharclass `\5 \digitClass
\XeTeXcharclass `\6 \digitClass
\XeTeXcharclass `\7 \digitClass
\XeTeXcharclass `\8 \digitClass
\XeTeXcharclass `\9 \digitClass

\newXeTeXintercharclass \dashClass
\XeTeXcharclass `\— \dashClass % em
\XeTeXcharclass `\– \dashClass % en
\XeTeXcharclass `\- \dashClass % hyphen
\XeTeXcharclass `\… \dashClass

\XeTeXinterchartokenstate = 1

\def\morhyphenpenalty{75}
\exhyphenpenalty=10000

\XeTeXinterchartoks \dashClass 0 = {\hskip 0pt\penalty \morhyphenpenalty}

% Adjust space around lists
\setlength{\topsep}{.5\baselineskip plus 1\baselineskip minus .5\baselineskip}
\setlength{\partopsep}{.5\baselineskip plus 1\baselineskip minus .5\baselineskip}

% Miscellaneous global typesetting parameters
\frenchspacing
\setlength{\emergencystretch}{.06\textwidth}

% Try to avoid excessive hyphens
\doublehyphendemerits=30000
\finalhyphendemerits=30000
\adjdemerits=10000
\brokenpenalty10000\relax

% Make it easier to manage hyphenation
\makeatletter
\newcommand{\emdashhyp}{\leavevmode%
\prw@zbreak—\discretionary{}{}{}\prw@zbreak}
\makeatother

% Avoid widows and orphans
% https://mailman.ntg.nl/pipermail/ntg-context/2013/074250.html
\widowpenalty 10000
\clubpenalty 10000

% For modulo and integer division operations
\usepackage{intcalc}

% For string replace operation
\usepackage{xstring}

\def\ReplaceDoubleHyphen#1{%
  \IfSubStr{#1}{--}{%
    \ReplaceDoubleHyphen{%
    \StrSubstitute{#1}{--}{-}}{#1}}
    }%

% Fake Esperanto NUMBERstring support in fmtcount
% https://tex.stackexchange.com/a/578200
\renewcommand{\NUMBERstring}[1]{\NUMBERstringnum{\the\value{#1}}}
\renewcommand{\NUMBERstringnum}[1]{\ReplaceDoubleHyphen{%
-\relax%
\ifcase\intcalcDiv{\intcalcMod{#1}{10000}}{1000}\relax% Case 0.
\or MIL% Case 1.
\or DUMIL% Case 2.
\or TRIMIL% Case 3.
\or KVARMIL% Case 4.
\or KVINMIL% Case 5.
\or SESMIL% Case 6.
\or SEPMIL% Case 7.
\or OKMIL% Case 8.
\or NAŬMIL% Case 9.
\fi%
-\relax%
\ifcase\intcalcDiv{\intcalcMod{#1}{1000}}{100}\relax% Case 0.
\or CENT% Case 1.
\or DUCENT% Case 2.
\or TRICENT% Case 3.
\or KVARCENT% Case 4.
\or KVINCENT% Case 5.
\or SESCENT% Case 6.
\or SEPCENT% Case 7.
\or OKCENT% Case 8.
\or NAŬCENT% Case 9.
\fi%
-\relax%
\ifcase\intcalcDiv{\intcalcMod{#1}{100}}{10}\relax% Case 0.
\or DEK% Case 1.
\or DUDEK% Case 2.
\or TRIDEK% Case 3.
\or KVARDEK% Case 4.
\or KVINDEK% Case 5.
\or SESDEK% Case 6.
\or SEPDEK% Case 7.
\or OKDEK% Case 8.
\or NAŬDEK% Case 9.
\fi%
-\relax%
\ifcase\intcalcMod{#1}{10}\relax% Case 0.
\or UNU% Case 1.
\or DU% Case 2.
\or TRI% Case 3.
\or KVAR% Case 4.
\or KVIN% Case 5.
\or SES% Case 6.
\or SEP% Case 7.
\or OK% Case 8.
\or NAŬ% Case 9.
\fi%
}}

% Various packages used
\usepackage[normalem]{ulem}
\usepackage{xfrac}
\usepackage{censor}
\usepackage[useregional]{datetime2}
\usepackage[nopagecolor=white,pagecolor=white]{pagecolor}
\usepackage{afterpage}
\usepackage{gitinfo2}
\usepackage{hyphenat}
\usepackage{comment}
\usepackage{hyphenat}
