% Logical markup

% These commands should be used to help make the source easy to understand
% and consistently typeset.
% Search for them in the source files to see how to use them.

% Special types of text
\newcommand{\abbrev}[1]{\textsc{\MakeLowercase{#1}}\xspace}

% Common abbreviations
\newcommand{\am}{~a.m.\xspace}
\renewcommand{\pm}{~p.m.\xspace}
\newcommand{\SPHEW}{\abbrev{SPHEW}}
% Tone of voice
\newcommand{\shout}[1]{\textsc{#1}}
\newcommand{\scream}[1]{\MakeUppercase{#1}}
\newcommand{\prophesy}[1]{\shout{#1}}

% parsel
\newcommand{\parselify}[1]{%
  \StrSubstitute{#1}{ss}{ß}[\parselstring]%
  \StrSubstitute{\parselstring}{s}{ss}[\parsselstring]%
  \StrSubstitute{\parsselstring}{ß}{sss}[\parssselstring]%
}
% N.B. Other commands, such as \emph, cannot be used inside \parsel
%
% parsel
% old version of parsel
\newcommand{\parsela}[1]{\parselify{#1}%
{\ptsansi\parssselstring}}
%
% new FR version of parsel by yeKcim
% note: Other commands, such as \emph, cannot be used inside \parsel
\newcommand{\parselb}[1]{\parselify{\fontspec[ExternalLocation]{Parseltongue.ttf}#1}%
{\ptsansi\parssselstring}}
%
% select one of above
\newcommand{\parsel}[1]{\parsela{#1}}

% \spell macro
\newcommand{\spell}[1]{{\Star}\emph{#1}{\Star}}

% Author’s notes
% \newcommand{\authorsnotefootnotemark}{\footnotemark}
% \newcommand{\authorsnotetext}[1]{\footnotetext{Noto de la aŭtoro: #1}}
% \newcommand{\translatorsnotefootnotemark}{\footnotemark}
% \newcommand{\translatorsnotetext}[1]{\footnotetext{Noto de la tradukisto: #1}}
\newcommand{\authorsnotefootnotemark}{}
\newcommand{\authorsnotetext}[1]{}
\newcommand{\translatorsnotefootnotemark}{}
\newcommand{\translatorsnotetext}[1]{}

% McGonagall's board
\newcommand{\McGonagallWhiteBoard}[1]{%
  \begin{center}
    \newsavebox{\hpbox}%
    \fontspec[ExternalLocation,Color=AA0000]{Florante}
    \savebox{\hpbox}{\MakeUppercase{#1}}
    \vspace{0.5ex}
    \usebox{\hpbox}
    \settowidth{\versewidth}{\usebox{\hpbox}}
    \vskip -1ex
    \fontspec[ExternalLocation,Color=2020FF]{ArchitectsDaughter}
    \resizebox{\versewidth}{.6ex}{\rotatebox{90}{I}}
  \end{center}%
}


% Newspaper headlines

\newcommand{\headline}[1]{%
\begin{center}%
\textsc{#1}%
\end{center}}

\newcommand{\inlineheadline}[1]{%
\textsc{#1}%
}

\newcommand{\newspaperHeader}[1]{\begin{SingleSpace}\upshape #1\end{SingleSpace}}
\newcommand{\newspaperName}[1]{\upshape\itshape #1}

\newenvironment{headlines}{%
  \begin{Spacing}{0.75}
  \begin{center}
  \scshape
  \nonzeroparskip
}{
  \end{center}
  \end{Spacing}
}


% Letters / writtenNote

\newenvironment{writtenNote}{%
\fontspec[ExternalLocation]{Graphe_Alpha_alt.ttf}\scriptsize%
\renewcommand{\emph}{\uline}%
\vskip .1\baselineskip plus .1\baselineskip minus .1\baselineskip%
\begin{adjustwidth}{\parindent}{\parindent}%
\par\setlength{\parindent}{0pt}\setlength{\parskip}{\baselineskip}%
\itshape%
}
{%
\par
\end{adjustwidth}%
\vskip 1\baselineskip plus 1\baselineskip minus 1\baselineskip%
}

% \letterAddress
\newcommand{\letterAddress}[1]{%
\pagebreak[1]\fontspec[ExternalLocation]{Graphe_Alpha_alt.ttf}%
\scriptsize#1\nopagebreak[4]\par%
}

% \letterClosing
\newcommand{\letterClosing}[2][\vskip 1\baselineskip]{%
\nopagebreak[4]\fontspec[ExternalLocation]{Graphe_Alpha_alt.ttf}%
\par\scriptsize#1%
\par\nopagebreak[5]#2%
}


% PartChapters
% \partchapter{The Stanford Prison Experiment}{TSPE}{XIII}{Aftermaths}
% TOC: TSPE part XIII: Aftermaths
% Page header: The Stanford Prison Experiment XIII: \\? Aftermaths
% Title: The Stanford Prison Experiment, Part XIII: \\? Aftermaths
\newcommand{\partchapter}[3][\relax]{%
	\chapter[\texorpdfstring{#2, \abbrev{part #3}}{#2, part #3}]%
		[#2 #3]{#2, Part~#3#1}}
\newcommand{\namedpartchapter}[5][\relax]{%
	\chapter[%
			\texorpdfstring{%
				\abbrev{#3, part #4}: #5}{%
				#3, part #4: #5}][%
			\mbox{#2 #4:} \mbox{#5}]{%
			#2, Part~#4:\protect\linebreak[1] #5#1}%
}

% Hanging paras for play scripts (used in Omake IV)
\newenvironment{playdialog}{\begin{hangparas}{2em}{1}}{\end{hangparas}}

% Chapter openings
% Definition of chapterOpeningAuthorNote when they are desired to be printed
% FIXME: Make the environment definition switchable with a flag
% \newenvironment{chapterOpeningAuthorNote}{%
% \par\noindent%
% E.~Y.:~
% }
% {%
% \newline\rule[1ex]{\textwidth}{.1pt}\newline%
% }
\excludecomment{chapterOpeningAuthorNote}

\newenvironment{chapterOpeningQuote}{%
\parindent=0pt%
\itshape}
{%
\\[1\baselineskip]%
%\newline%
%\rule[1ex]{\textwidth}{.1pt}\newline%x
}

% Stars and breaks

% Single “magic star” decoration
\newcommand{\Star}{{\fontspec[ExternalLocation]{Miscelanea.ttf}*}}

% Three “magic stars” decoration
\def\Stars{{\large\Star\kern-.6ex\lower1.3ex\hbox{\large\Star}\kern-.1ex\raise.2ex\hbox{\tiny\Star}\spacefactor1000}}

% Text break made of \Stars (only used to define other commands)
\makeatletter
\def\sbre@k{\mbox{}\nobreak\hfill\mbox{}\allowbreak\rule{.60\textwidth}{.0pt}\par%
  \vskip 0pt plus 2\baselineskip\noindent{%
    \parbox[c][0pt][c]{\textwidth}{%
      \hfil \hbox{\lower14pt\hbox{\normalsize\Stars}}%
    }%
  }}

% A standalone break
\def\later{\sbre@k%
  \vskip 0pt plus 2\baselineskip%
  \par\rule{.5\textwidth}{.0pt}\vskip1pt\noindent}

% A break followed by a new section
\newcommand{\latersection}[1]{\sbre@k\section{#1}}
\makeatother
