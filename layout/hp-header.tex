\usepackage{lettrine}		% Used for the fancy caps at each start of each chapter.
\usepackage{xspace}		% Takes care of spaces after macros
\usepackage{amsmath}		% Provides the align environment, used in chapter 13 for the notes

\usepackage[protrusion=true]{microtype}

\usepackage{fontspec}		% For the many fonts
\usepackage{xunicode}

\usepackage{xstring}

\usepackage{eso-pic,picture}

\usepackage[
	bookmarks=true,
	unicode=true,
	pdfborder={0 0 0},
	pdftitle={Hari Potr kaj la Metodoj de Racionaleco},
	pdfauthor={Eliezer Judkaŭski},
 	breaklinks={true},
	pdfkeywords={Harry Potter, Hari Potr, rationality, racionaleco, hpmor, esperanto, translation},
	pdfencoding=auto
]{hyperref}

% This file includes all the generic formatting for HPatMoR. This mostly entails configuring
% the memoir package, though “configuring” on occasion means “completely messing it up”.

\RequirePackage{fmtcount}
\RequirePackage{calc}

% Fonts used generally (specific fonts used only once or twice are not here).
\usepackage{xltxtra}
\defaultfontfeatures{Ligatures={TeX}}
\setmainfont[
  Extension=.otf
, UprightFont=*-Regular
, ItalicFont=*-Italic
, BoldFont=*-Bold
, BoldItalicFont=*-BoldItalic
, SmallCapsFont=AlegreyaSC-Regular
]{Alegreya}

\newfontface\hpfont[ExternalLocation]{Lumos}
\newfontface\ptsansi[ExternalLocation]{AlegreyaSans-Italic}

% Drop-caps at start of chapters
\renewcommand{\LettrineFontHook}{\hpfont}
\renewcommand{\LettrineTextFont}{}
\renewcommand{\DefaultLoversize}{.2}
\renewcommand{\DefaultLraise}{0.1}

% Extend lettrine cutout when more than one paragraph goes alongside the drop-cap
% Copied, with different macro names, from
% https://tex.stackexchange.com/questions/369868/using-lettrine-with-short-paragraphs
\newcount\hplettrinecount
\makeatletter
\def\hplettrineextrapara{%
\ifnum\prevgraf<\c@L@lines
\hplettrinecount\z@
\loop
\ifnum\hplettrinecount<\prevgraf
\advance\hplettrinecount\@ne
\afterassignment\hplettrinedima\count@\L@parshape\relax
\repeat
\parshape\L@parshape
\fi}
\def\hplettrinedima{\afterassignment\hplettrinedimb\dimen@}
\def\hplettrinedimb{\afterassignment\hplettrinedef\dimen@}
\def\hplettrinedef#1\relax{\edef\L@parshape{\the\numexpr\count@-1\relax\space #1}}
\makeatother
\newcommand{\lettrinepara}[3][]{\lettrine[nindent=0pt,#1]{#2}{#3}}

% Allow linebreaks after em-dash and hyphens, except when they’re followed by punctuation
\newXeTeXintercharclass \punctuationClass

\XeTeXcharclass `\’ \punctuationClass
\XeTeXcharclass `\‘ \punctuationClass
\XeTeXcharclass `\“ \punctuationClass
\XeTeXcharclass `\” \punctuationClass
\XeTeXcharclass `\. \punctuationClass
\XeTeXcharclass `\, \punctuationClass
\XeTeXcharclass `\: \punctuationClass
\XeTeXcharclass `\? \punctuationClass
\XeTeXcharclass `\! \punctuationClass
\XeTeXcharclass `\: \punctuationClass

\newXeTeXintercharclass \digitClass
\XeTeXcharclass `\0 \digitClass
\XeTeXcharclass `\1 \digitClass
\XeTeXcharclass `\2 \digitClass
\XeTeXcharclass `\3 \digitClass
\XeTeXcharclass `\4 \digitClass
\XeTeXcharclass `\5 \digitClass
\XeTeXcharclass `\6 \digitClass
\XeTeXcharclass `\7 \digitClass
\XeTeXcharclass `\8 \digitClass
\XeTeXcharclass `\9 \digitClass

\newXeTeXintercharclass \dashClass
\XeTeXcharclass `\— \dashClass % em
\XeTeXcharclass `\– \dashClass % en
\XeTeXcharclass `\- \dashClass % hyphen
\XeTeXcharclass `\… \dashClass

\XeTeXinterchartokenstate = 1

\def\morhyphenpenalty{75}
\exhyphenpenalty=10000

\XeTeXinterchartoks \dashClass 0 = {\hskip 0pt\penalty \morhyphenpenalty}

% Adjust space around lists
\setlength{\topsep}{.5\baselineskip plus 1\baselineskip minus .5\baselineskip}
\setlength{\partopsep}{.5\baselineskip plus 1\baselineskip minus .5\baselineskip}

% Miscellaneous global typesetting parameters
\frenchspacing
\setlength{\emergencystretch}{.06\textwidth}

% Try to avoid excessive hyphens
\doublehyphendemerits=30000
\finalhyphendemerits=30000
\adjdemerits=10000
\brokenpenalty10000\relax

% Make it easier to manage hyphenation
\makeatletter
\newcommand{\emdashhyp}{\leavevmode%
\prw@zbreak—\discretionary{}{}{}\prw@zbreak}
\makeatother

% Avoid widows and orphans
% https://mailman.ntg.nl/pipermail/ntg-context/2013/074250.html
\widowpenalty 10000
\clubpenalty 10000

% For modulo and integer division operations
\usepackage{intcalc}

% Fake Esperanto NUMBERstring support in fmtcount
% https://tex.stackexchange.com/a/578200
\renewcommand{\NUMBERstringnum}[1]{%
\ifcase\intcalcDiv{\intcalcMod{#1}{10000}}{1000}\relax% Case 0.
\or MIL% Case 1.
\or DUMIL% Case 2.
\or TRIMIL% Case 3.
\or KVARMIL% Case 4.
\or KVINMIL% Case 5.
\or SESMIL% Case 6.
\or SEPMIL% Case 7.
\or OKMIL% Case 8.
\or NAŬMIL% Case 9.
\fi%
\ifcase\intcalcDiv{\intcalcMod{#1}{1000}}{100}\relax% Case 0.
\or CENT% Case 1.
\or DUCENT% Case 2.
\or TRICENT% Case 3.
\or KVARCENT% Case 4.
\or KVINCENT% Case 5.
\or SESCENT% Case 6.
\or SEPCENT% Case 7.
\or OKCENT% Case 8.
\or NAŬCENT% Case 9.
\fi%
\ifcase\intcalcDiv{\intcalcMod{#1}{100}}{10}\relax% Case 0.
\or DEK% Case 1.
\or DUDEK% Case 2.
\or TRIDEK% Case 3.
\or KVARDEK% Case 4.
\or KVINDEK% Case 5.
\or SESDEK% Case 6.
\or SEPDEK% Case 7.
\or OKDEK% Case 8.
\or NAŬDEK% Case 9.
\fi ~%
\ifcase\intcalcMod{#1}{10}\relax% Case 0.
\or UNU% Case 1.
\or DU% Case 2.
\or TRI% Case 3.
\or KVAR% Case 4.
\or KVIN% Case 5.
\or SES% Case 6.
\or SEP% Case 7.
\or OK% Case 8.
\or NAŬ% Case 9.
\fi%
}

% Various packages used
\usepackage[normalem]{ulem}
\usepackage{xfrac}
\usepackage{censor}
\usepackage[useregional]{datetime2}
\usepackage[nopagecolor=white,pagecolor=white]{pagecolor}
\usepackage{afterpage}
\usepackage{gitinfo2}
\usepackage{hyphenat}
\usepackage{comment}
\usepackage{hyphenat}

% Logical markup

% These commands should be used to help make the source easy to understand
% and consistently typeset.
% Search for them in the source files to see how to use them.

% Special types of text
\newcommand{\abbrev}[1]{\textsc{\MakeLowercase{#1}}\xspace}

% Common abbreviations
\newcommand{\am}{~a.m.\xspace}
\renewcommand{\pm}{~p.m.\xspace}
\newcommand{\SPHEW}{\abbrev{SPHEW}}
% Tone of voice
\newcommand{\shout}[1]{\textsc{#1}}
\newcommand{\scream}[1]{\MakeUppercase{#1}}
\newcommand{\prophesy}[1]{\shout{#1}}

% parsel
\newcommand{\parselify}[1]{%
  \StrSubstitute{#1}{ss}{ß}[\parselstring]%
  \StrSubstitute{\parselstring}{s}{ss}[\parsselstring]%
  \StrSubstitute{\parsselstring}{ß}{sss}[\parssselstring]%
}
% N.B. Other commands, such as \emph, cannot be used inside \parsel
%
% parsel
% old version of parsel
\newcommand{\parsela}[1]{\parselify{#1}%
{\ptsansi\parssselstring}}
%
% new FR version of parsel by yeKcim
% note: Other commands, such as \emph, cannot be used inside \parsel
\newcommand{\parselb}[1]{\parselify{\fontspec[ExternalLocation]{Parseltongue.ttf}#1}%
{\ptsansi\parssselstring}}
%
% select one of above
\newcommand{\parsel}[1]{\parsela{#1}}

% \spell macro
\newcommand{\spell}[1]{{\Star}\emph{#1}{\Star}}

% Author’s notes
\newcommand{\authorsnotefootnotemark}{\footnotemark}
\newcommand{\authorsnotetext}[1]{\footnotetext{Noto de la aŭtoro: #1}}
\newcommand{\translatorsnotefootnotemark}{\footnotemark}
\newcommand{\translatorsnotetext}[1]{\footnotetext{Noto de la tradukisto: #1}}

% McGonagall's board
\newcommand{\McGonagallWhiteBoard}[1]{%
  \begin{center}
    \newsavebox{\hpbox}%
    \fontspec[ExternalLocation,Color=AA0000]{Florante}
    \savebox{\hpbox}{\MakeUppercase{#1}}
    \vspace{0.5ex}
    \usebox{\hpbox}
    \settowidth{\versewidth}{\usebox{\hpbox}}
    \vskip -1ex
    \fontspec[ExternalLocation,Color=2020FF]{ArchitectsDaughter}
    \resizebox{\versewidth}{.6ex}{\rotatebox{90}{I}}
  \end{center}%
}


% Newspaper headlines

\newcommand{\headline}[1]{%
\begin{center}%
\textsc{#1}%
\end{center}}

\newcommand{\inlineheadline}[1]{%
\textsc{#1}%
}

\newcommand{\newspaperHeader}[1]{\begin{SingleSpace}\upshape #1\end{SingleSpace}}
\newcommand{\newspaperName}[1]{\upshape\itshape #1}

\newenvironment{headlines}{%
  \begin{Spacing}{0.75}
  \begin{center}
  \scshape
  \nonzeroparskip
}{
  \end{center}
  \end{Spacing}
}


% Letters / writtenNote

\newenvironment{writtenNote}{%
\fontspec[ExternalLocation]{Graphe_Alpha_alt.ttf}\scriptsize%
\renewcommand{\emph}{\uline}%
\vskip .1\baselineskip plus .1\baselineskip minus .1\baselineskip%
\begin{adjustwidth}{\parindent}{\parindent}%
\par\setlength{\parindent}{0pt}\setlength{\parskip}{\baselineskip}%
\itshape%
}
{%
\par
\end{adjustwidth}%
\vskip 1\baselineskip plus 1\baselineskip minus 1\baselineskip%
}

% \letterAddress
\newcommand{\letterAddress}[1]{%
\pagebreak[1]\fontspec[ExternalLocation]{Graphe_Alpha_alt.ttf}%
\scriptsize#1\nopagebreak[4]\par%
}

% \letterClosing
\newcommand{\letterClosing}[2][\vskip 1\baselineskip]{%
\nopagebreak[4]\fontspec[ExternalLocation]{Graphe_Alpha_alt.ttf}%
\par\scriptsize#1%
\par\nopagebreak[5]#2%
}


% PartChapters
% \partchapter{The Stanford Prison Experiment}{TSPE}{XIII}{Aftermaths}
% TOC: TSPE part XIII: Aftermaths
% Page header: The Stanford Prison Experiment XIII: \\? Aftermaths
% Title: The Stanford Prison Experiment, Part XIII: \\? Aftermaths
\newcommand{\partchapter}[3][\relax]{%
	\chapter[\texorpdfstring{#2, \abbrev{part #3}}{#2, part #3}]%
		[#2 #3]{#2, Part~#3#1}}
\newcommand{\namedpartchapter}[5][\relax]{%
	\chapter[%
			\texorpdfstring{%
				\abbrev{#3, part #4}: #5}{%
				#3, part #4: #5}][%
			\mbox{#2 #4:} \mbox{#5}]{%
			#2, Part~#4:\protect\linebreak[1] #5#1}%
}

% Hanging paras for play scripts (used in Omake IV)
\newenvironment{playdialog}{\begin{hangparas}{2em}{1}}{\end{hangparas}}

% Chapter openings
% Definition of chapterOpeningAuthorNote when they are desired to be printed
% FIXME: Make the environment definition switchable with a flag
% \newenvironment{chapterOpeningAuthorNote}{%
% \par\noindent%
% E.~Y.:~
% }
% {%
% \newline\rule[1ex]{\textwidth}{.1pt}\newline%
% }
\excludecomment{chapterOpeningAuthorNote}

\newenvironment{chapterOpeningQuote}{%
\parindent=0pt%
\itshape}
{%
\\[1\baselineskip]%
%\newline%
%\rule[1ex]{\textwidth}{.1pt}\newline%x
}

% Stars and breaks

% Single “magic star” decoration
\newcommand{\Star}{{\fontspec[ExternalLocation]{Miscelanea.ttf}*}}

% Three “magic stars” decoration
\def\Stars{{\large\Star\kern-.6ex\lower1.3ex\hbox{\large\Star}\kern-.1ex\raise.2ex\hbox{\tiny\Star}\spacefactor1000}}

% Text break made of \Stars (only used to define other commands)
\makeatletter
\def\sbre@k{\mbox{}\nobreak\hfill\mbox{}\allowbreak\rule{.60\textwidth}{.0pt}\par%
  \vskip 0pt plus 2\baselineskip\noindent{%
    \parbox[c][0pt][c]{\textwidth}{%
      \hfil \hbox{\lower14pt\hbox{\normalsize\Stars}}%
    }%
  }}

% A standalone break
\def\later{\sbre@k%
  \vskip 0pt plus 2\baselineskip%
  \par\rule{.5\textwidth}{.0pt}\vskip1pt\noindent}

% A break followed by a new section
\newcommand{\latersection}[1]{\sbre@k\section{#1}}
\makeatother

% Toggle IsEnglish for debugging
%
%     \iftoggle{isEnglish}{%
%         \lettrine{E}{very} inch of wall space is covered by a bookcase.
%     }{%
%         \lettrine{Ĉ}{iu} colo de muro-spaco estas kovrita per libro-ŝrankaĵo.
%     }
%
% See: https://tex.stackexchange.com/a/5896
\newtoggle{isEnglish}


\hyphenation{Her-maj-ni Gran-ger Gri-fin-dor Le-stranĝ
Hog-ŭartso Wi-zen-gam-ot Sli-te-ren Sli-te-renoj Sli-te-renojn
Se-ve-rus Mik-gon-agol Dum-bel-dor Kŭi-rel Mal-foj}


% Redefine \textls for XeTeX
\usepackage{calc}
\newcounter{hpletterspacing}
\renewcommand{\textls}[2][100]{%
  \setcounter{hpletterspacing}{#1 / \real{10.0}}%
  \addfontfeature{LetterSpace=\thehpletterspacing}#2%
}
